  \documentclass[10pt,a4paper]{article}

\usepackage{amsmath}   


\begin{document}
\section{Constant Velocity Model}
Der Zustand  $\underline{x}_{k-1}$ an dem Zeitschritt $k-1$, der die Bewegung des Objekts beschreibt, erhält die Postionen $x_{k-1}$ und $y_{k-1}$ und die Geschwindigkeiten $v^x_{k-1}$ beziehungsweise $v^y_{k-1}$.
\begin{equation}
\underline{x}_{k-1} = \begin{bmatrix}
x_{k-1} \\
v^x_{k-1}\\
y_{k-1}\\
v^y_{k-1}\\
\end{bmatrix}.
\end{equation}
$x_{k-1}$ und $y_{k-1}$ können gemessen werden und $v^x_{k-1}$ und $v^y_{k-1}$ werden geschätzt, somit:
\begin{equation}
\underline{y}_{k-1} = \begin{bmatrix}
x_{k-1} \\
y_{k-1}\\
\end{bmatrix} =\textbf{H}\underline{x}_{k-1} =
\begin{bmatrix}
1 & 0 &0 &0 \\
0 & 0 &1 &0\\
\end{bmatrix}\underline{x}_{k-1},
\end{equation}
die Matrix $\textbf{H}$ wird als \textit{Ausgangsmatrix} betrachtet. Es wird angenommen, dass die Objekte sich mit näherungsweise eine konstante Geschwindigkeit bewegen. Aus dem Grund, können die Postionen und Geschwindigkeiten folgendermaßen beschrieben werden.
\begin{equation}
x_{k} = x_{k-1} + Tv^x_{k-1},
\end{equation}
\begin{equation*}
y_{k} = y_{k-1} + Tv^y_{k-1},
\end{equation*}
\begin{equation*}
v^x_{k} = v^x_{k-1},
\end{equation*}
\begin{equation*}
v^y_{k} = v^y_{k-1},
\end{equation*}
wo $T$ die Bildaufnahmezeit darstellt. Zusammenfassend stellt man das diskrete dynamische System vor:
\begin{equation}
\underline{x}_{k} = \textbf{F}\underline{x}_{k-1} = 
\begin{bmatrix}
1 & T &0 &0 \\
0 & 1 &0 &0 \\
0 & 0 &1 &T\\
0 & 0 &0 &1 \\
\end{bmatrix}\underline{x}_{k-1},
\end{equation}
wo die Matrix $\textbf{F}$ die diskrete Systemmatrix darstellt.


\end{document}
